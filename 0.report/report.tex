\documentclass[12pt]{article}
 
%Russian-specific packages
%--------------------------------------
\usepackage[T2A]{fontenc}
\usepackage[utf8]{inputenc}
\usepackage[english, russian]{babel}
%--------------------------------------

%Math-specific packages
%--------------------------------------
\usepackage{amsmath}
\usepackage{amssymb}
%--------------------------------------

\title{Нахождение корней 
    линейной системы уравнения
    с помощью метода Гаусса с 
    нахождением максимального элемента 
    по столбцу} 
\author{Шерстобитов Андрей \\
         332 группа}
\begin{document}
    \maketitle
\newpage    
    \tableofcontents
    
\newpage
    \section{Постановка задачи}

    \quad Пусть дана матрица $A$ размера $n \times n$, вектор $b$ размера $n$. 
    Требуется найти решение линейной системы $Ax = b$, используя метод
    Гаусса с выбором максимального элмента по столбцу.
    
    \section{Алгоритм}
        \subsection{Разбиение матрицы на блоки}
        \quad Разобьем матрицу $A$ на блоки размера $m \times m$ 
        по формуле $ n = k \cdot m + l $:
        \[
        \left(
        \begin{array}{ccccc|c}
            A_{1,1}^{m \times m}   & A_{1,2}^{m \times m}   & \ldots & A_{1,k}^{m \times m}   & A_{1,k+1}^{m \times l}    & B^{m}_{1}   \\
            A_{2,1}^{m \times m}   & A_{2,2}^{m \times m}   & \ldots & A_{2,k}^{m \times m}   & A_{2,k+1}^{m \times l}    & B^{m}_{2}   \\
            \ldots                 & \ldots                 & \ldots & \ldots                 & \ldots                    & \ldots      \\
            A_{k,1}^{m \times m}   & A_{k,2}^{m \times m}   & \ldots & A_{k,k}^{m \times m}   & A_{k,k+1}^{m \times l}    & B^{m}_{k}   \\
            A_{k+1,1}^{l \times m} & A_{k+1,2}^{l \times m} & \ldots & A_{k+1,k}^{l \times m} & A_{k+1, k+1}^{l \times l} & B^{l}_{k+1} \\
        \end{array}
        \right)
        \]
        
        \subsection{Прямой ход}

        \begin{enumerate}
            
            \item \label{alg:reverse} $ A^{m \times m}_{1,1} \rightarrow V_{1} $, методом Гаусса с выбором
            максимального элемента по столбцу для находим обратную матрицу $V_{1} | V_{3} \Rightarrow V_{3} = V_{1}^{-1}
            =  (A^{m \times m}_{1,1})^{-1}$, $min = \| V_{3} \|  $ 
            
            Если $ \nexists (A^{m \times m}_{1,1})^{-1}$, то выбираем следующую строчку.
            
            Если $\forall A^{m \times m}_{i,1}, i = 1, \ldots k \ \nexists (A^{m \times m}_{i,1})^{-1}$,
            то алгоритм не применим.
            
            $\forall i = 2, \ldots k \ A^{m \times m}_{i,1} \rightarrow V_{1}$.
            $V_{1} | V_{2} \Rightarrow V_{2} = V_{1}^{-1}$, если $\| V_{2} \| < min$, 
            то меняем местами указатели $V_{3}$ и $V_{2}$.   
            
            \item \label{alg:swap} $I_{i} \leftrightarrow I_{1}$ (строчки)
            
            \item \label{alg:mult} 
            $A_{1,1}^{m \times m} = E_{1,1}^{m \times m}, \ V_{3} \times (A_{1,2}^{m \times m} \ \ldots \ A_{1,k}^{m \times m} \ A_{1,k+1}^{m \times l}
             = B^{m}_{1})$: \\
            $A_{1,j}^{m \times m}, j = 2, \ldots, k; \ A_{1,j}^{m \times l}, j = k + 1$ сохраняем в $V_{1}$,
            затем $V_{2} = V_{3} \cdot V_{1}$ и $V_{2} \rightarrow A_{1,j}$.
            В итоге получаем: $ E_{1,1}^{m \times m} \ A_{1,2}^{m \times m^*}    \ \ldots \ A_{1,k}^{m \times m^*} \ A_{1,k+1}^{m \times l^*} = B^{m^*}_{1}$
            
            \item  \label{alg:form} Далее действуем по формуле $A_{i, j} = A_{i, j} - A_{i,1} \cdot A_{1,j}$, получим матрицу:
            \[
            \left(
            \begin{array}{ccccc|c}
                E_{1,1}^{m \times m} & A_{1,2}^{m \times m^*}   & \ldots & A_{1,k}^{m \times m^*}   & A_{1,k+1}^{m \times l^*}    & B^{m^*}_{1}   \\  
                0                    & A_{2,2}^{m \times m^*}   & \ldots & A_{2,k}^{m \times m^*}   & A_{2,k+1}^{m \times l^*}    & B^{m^*}_{2}   \\  
                \ldots               & \ldots                   & \ldots & \ldots                   & \ldots                      & \ldots        \\  
                0                    & A_{k,2}^{m \times m^*}   & \ldots & A_{k,k}^{m \times m^*}   & A_{k,k+1}^{m \times l^*}    & B^{m^*}_{k}   \\  
                0                    & A_{k+1,2}^{l \times m^*} & \ldots & A_{k+1,k}^{l \times m^*} & A_{k+1, k+1}^{l \times l^*} & B^{l^*}_{k+1} \\  
            \end{array}
            \right)
            \]

            \item \label{alg:repeat} Далее повторяем алгоритм для матрицы $(n - m) \times (n - m)$:
            \[
            \left(
            \begin{array}{cccc|c}
            A_{2,2}^{m \times m^*}     & \ldots & A_{2,k}^{m \times m^*}   & A_{2,k+1}^{m \times l^*}       & B^{m^*}_{2}   \\  
            \ldots                     & \ldots &                          & \ldots                         & \ldots        \\  
            A_{k,2}^{m \times m^*}     & \ldots & A_{k,k}^{m \times m^*}   & A_{k,k+1}^{m \times l^*}       & B^{m^*}_{k}   \\  
            A_{k+1,2}^{l \times m^*}   & \ldots & A_{k+1,k}^{l \times m^*} & A_{k+1, k+1}^{l \times l^*}    & B^{l^*}_{k+1} \\
            \end{array}
            \right)
            \]
        
        \end{enumerate}
            
            На $r < k + 1$ ходу алгоритма: 
        \begin{enumerate}
            \item Ищем вышеописанную обратную среди матриц $A_{qr}^{m \times m},\ q = r, \ldots, k$ 
            \item $I_{r} \leftrightarrow I_{q}$ (строчки)
            \item $A_{r,r}^{m \times m} = E_{r,r}^{m \times m}, \ (A_{r,r}^{{m \times m}})^{-1} \times (A_{r,r+1}^{m \times m} \ \ldots \ A_{r,k}^{m \times m} \ A_{r,k+1}^{m \times l}
            = B_{r}^{m})$
            \item 
            $\begin{aligned}[t]
                i, j &= r+1, \ldots, k &\ A_{i, j}^{m \times m}   &= A_{i, j}^{m \times m} - A_{i,r}^{m \times m} \times A_{r,j}^{m \times m}        \\
                i   &= r+1, \ldots, k &\ A_{i,k+1}^{m \times l}   &= A_{i,k+1}^{m \times l} - A_{i,r}^{m \times m} \times A_{r,k+1}^{m \times l}     \\
                j   &= r+1, \ldots, k &\ A_{k+1,j}^{l \times m}   &= A_{k+1,j}^{l \times m} - A_{k+1,r}^{l \times m} \times A_{r,j}^{m \times m}     \\
                    &                 &\ A_{k+1,k+1}^{l \times l} &= A_{k+1,k+1}^{l \times l} - A_{k+1,r}^{l \times m} \times A_{r,k+1}^{m \times l} \\
                i   &= r+1, \ldots, k &\ B_{i}^{m}                &= B_{i}^{m} - A_{i,r}^{m \times m} \times B_{r}^{m} \\
                    &                 &\ B_{k+1}^{l}              &= B_{k+1}^{l} - A_{k+1,r}^{l \times m} \times B_{r}^{m}\\                
                i   &= r+1, \ldots, k &\ A_{i,r}^{m \times m}     &= 0 \\
                    &                 &\ A_{k+1,r}^{l \times m}   &= 0 
            \end{aligned}$
        \end{enumerate}

            На $r = k + 1$ ходу алгоритма:
            \begin{enumerate}
                \item \addtocounter{enumi}{1} Ищем вышеописанную обратную к $A_{k+1,k+1}^{l \times l}$
                \item $A_{k+1,k+1}^{l \times l} = E_{k+1,k+1}^{l \times l}, \ B_{k+1}^{l} = (A_{k+1,k+1}^{l \times l})^{-1} \times B_{k+1}^{l} $
            \end{enumerate}

            После прямого хода получаем матрицу:
            \[
            \left(
            \begin{array}{ccccc|c}
                E_{1,1}^{m \times m} & A_{1,2}^{m \times m^*} & \ldots & A_{1,k}^{m \times m^*} & A_{1,k+1}^{m \times l^*}    & B^{m^*}_{1}   \\  
                0                    & E_{2,2}^{m \times m^*} & \ldots & A_{2,k}^{m \times m^*} & A_{2,k+1}^{m \times l^*}    & B^{m^*}_{2}   \\  
                \ldots               & \ldots                 & \ldots & \ldots                 & \ldots                      & \ldots        \\  
                0                    & 0                      & \ldots & E_{k,k}^{m \times m^*} & A_{k,k+1}^{m \times l^*}    & B^{m^*}_{k}   \\  
                0                    & 0                      & \ldots & 0                      & E_{k+1, k+1}^{l \times l^*} & B^{l^*}_{k+1} \\  
            \end{array} 
            \right)
            \]

        \subsection{Обратный ход}
        
        \[r = 1: \ E_{k + 1, k + 1}^{l \times l} = X_{k+1}^{l}\]    
        \[r = 2: \ X_{k}^{m} = B_{i}^{k} - A^{m \times l}_{k,k+1} \cdot B^{l}_{k+1},\ A^{m \times l}_{k,k+1} = 0 \]
        \[
        \left(
        \begin{array}{ccccc|c}
            E_{1,1}^{m \times m} & A_{1,2}^{m \times m^*} & \ldots & A_{1,k}^{m \times m^*} & A_{1,k+1}^{m \times l^*}  & B^{m}_{1}   \\  
            0                    & E_{2,2}^{m \times m}   & \ldots & A_{2,k}^{m \times m^*} & A_{2,k+1}^{m \times l^*}  & B^{m}_{2}   \\  
            \ldots               & \ldots                 & \ldots & \ldots                 & \ldots                    & \ldots      \\  
            0                    & 0                      & \ldots & E_{k,k}^{m \times m}   & 0                         & X^{m}_{k}   \\  
            0                    & 0                      & \ldots & 0                      & E_{k+1, k+1}^{l \times l} & X^{l}_{k+1} \\  
        \end{array}
        \right)
        \]

        \[ \forall r = 3, \ldots, k:\ X_{i}^{m} = B_{i}^{m} - \sum_{j = i + 1}^{n} A^{m \times h}_{i, j} \cdot X^{h}_{j},\ A^{m \times h}_{i, j} = 0, i = k - 2, \ldots, 0\]
        \[h = (j\ <\ k )\ ?\ m\ :\ l \]
        \[
        \left(
        \begin{array}{ccccc|c}
            E_{1,1}^{m \times m} & 0                    & \ldots & 0                      & 0                         & X^{m}_{1}   \\  
            0                    & E_{2,2}^{m \times m} & \ldots & 0                      & 0                         & X^{m}_{2}   \\  
            \ldots               & \ldots               & \ldots & \ldots                 & \ldots                    & \ldots      \\  
            0                    & 0                    & \ldots & E_{k,k}^{m \times m  } & 0                         & X^{m}_{k}   \\  
            0                    & 0                    & \ldots & 0                      & E_{k+1, k+1}^{l \times l} & X^{l}_{k+1} \\  
        \end{array}
        \right)
        \]
    
        \section{Хранение в памяти}
    \quad Матрицу $A$ храним в памяти таким образом: 
        \begin{align*}
            A = \{&a_{1,1}, a_{1,2},  \ldots , a_{1,m},
                    a_{2,1}, a_{2,2}, \ldots , a_{2,m}, \ldots,
                    a_{m,1}, a_{m,2}, \ldots , a_{m,m}, \\
                    &a_{1,m+1}, a_{1,m+2}, \ldots , a_{1,2m}, \ldots, a_{n,n}, 
                    b_{1},\ldots, b_{n} \}
        \end{align*}
        \quad    В блочном виде:
        \begin{align*}
            A = \{&A_{1,1}, A_{1,2},\ldots , A_{1,k+1},
            A_{2,1}, A_{2,2},      \ldots , A_{2,k+1}, \ldots, \\
            &A_{k+1,1}, A_{k+1,2},  \ldots , A_{k+1,k+1},B_{1},\ldots, B_{k+1} \}
        \end{align*}

        Тогда указатель на блок $A_{i, j}$ $: A + (i - 1) \cdot n \cdot m + (j - 1) \cdot last\_i \cdot m,$. \\
        Указатель на элемент $(p,q)$ этой же матрицы: $A + (i - 1) \cdot n \cdot m + (j - 1) \cdot last\_i \cdot m + (p - 1) \cdot last\_j + (q - 1)$,
        где $ last\_i = i\ <\ k\ ?\ m\ :\ l,\ last\_j = j\ <\ k\ ?\ m\ :\ l$.
        
        Указатель на $i$-ый элемент $B$: $A + n \cdot n + (i - 1)$

        \section{Оценка сложности}
        \begin{enumerate}
            \item Нахождение обратной матрицы для $A^{n \times n}$: $\frac{8}{3}n^3 + O(n^2)$
            \item Умножение матриц $A^{n \times m} \times A^{m \times l}$: $2mnl - nl$
            \item Сложение матриц $A^{n \times m} + A^{n \times m}$: $nm$
        \end{enumerate}

        \subsection{Прямой ход}
        
        На $r < k + 1$ ходу прямого хода:

        На шаге \ref{alg:reverse}: \par
            \quad $(k - r + 1)$ нахождений обратной матрицы к матрице $m \times m$
        
        На шаге \ref{alg:mult}: \par
            \quad $(k - r)$ умножений матриц $m \times m$ на $m \times m$  \par
            \quad 1 умножение матрицы $m \times m$ на матрицу $m \times l$ \par
            \quad 1 умножение матрицы $m \times m$ на матрицу $m \times 1$

        На шаге \ref{alg:form}: \par
            \quad $m \times m$ : по $(k - r)^2$ умножений и сложений        \par
            \quad по $(k - r)$ умножений матриц $l \times m$ на $m \times m$
                                              и $m \times m$ на $m \times l$ \par
            \quad по $(k - r)$ сложений матриц $l \times m$ и $m \times l$   \par
            \quad $(k - r)$ умножений матрицы $m \times m$ на $m \times 1$   \par
            \quad $(k - r)$ сложений матриц $m \times 1$                     \par
            \quad 1 умножение матрицы $l \times m$ на $m \times l$           \par
            \quad 1 сложение $l \times l$                                    \par
            \quad 1 умножение $l \times m$ матрицы на $m \times 1$           \par
            \quad 1 сложение $l \times 1$ матриц                             \\
        На $r = k + 1$ ходу:    

        На шаге \ref{alg:reverse}: \par
            \quad 1 нахождениe обратной матрицы к матрице $l \times l$
        
        На шаге \ref{alg:mult}: \par
            \quad 1 умножение матрицы $l \times l$ на матрицу $l \times 1$

            Сложнось прямого хода:
        \begin{multline*}
            \sum_{i=0}^{k-1} ((i + 1) \frac{8}{3}m^3 + i(i + 1) (2m^3 - m^2) + i^2m^2 + (1+2i)(2m^2l-ml) + 2iml \\ + 2ml^2 - ml + l^2 + (1 + i)(2m^2-m) + im + 2ml - l + l) + \frac{8}{3}l^3+2l^2-l
        \end{multline*}
        \subsection{Обратный ход}
        На шаге 1: $k$ умножений матриц $m \times l$ на $l \times 1$ и $k$ сложений $m \times 1$ матриц. \\
        На каждом следующем шаге $r$: $(k - r)$ умножений $m \times m$ на $m \times 1$ и $(k - r)$ сложений $m \times 1$
        матриц.

        Сложность обратного хода:
        \[ \sum^{k-2}_{i=0} (i(2m^2-m) + im) + k(ml - m) + km \]

        \subsection{Сложность алгоритма}
        Сложность всего алгоритма: 
        \[\boxed{\frac{2}{3}n^3 + \frac{4}{3}mn^2 + \frac{2}{3}m^2n + O(n^2)} \]

        При $m = 1$: $\frac{2}{3}n^3 + O(n^2)$;

        При $m = n$: $\frac{8}{3}n^3 + O(n^2)$;
\end{document}